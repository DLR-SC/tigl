%% Based on a TeXnicCenter-Template by Tino Weinkauf.
%%%%%%%%%%%%%%%%%%%%%%%%%%%%%%%%%%%%%%%%%%%%%%%%%%%%%%%%%%%%%

%%%%%%%%%%%%%%%%%%%%%%%%%%%%%%%%%%%%%%%%%%%%%%%%%%%%%%%%%%%%%
%% HEADER
%%%%%%%%%%%%%%%%%%%%%%%%%%%%%%%%%%%%%%%%%%%%%%%%%%%%%%%%%%%%%
\documentclass[a4paper,twoside,10pt]{report}
% Alternative Options:
%	Paper Size: a4paper / a5paper / b5paper / letterpaper / legalpaper / executivepaper
% Duplex: oneside / twoside
% Base Font Size: 10pt / 11pt / 12pt


%% Language %%%%%%%%%%%%%%%%%%%%%%%%%%%%%%%%%%%%%%%%%%%%%%%%%
\usepackage[english]{babel}	% put only one language here
\selectlanguage{english}			% f�r silbentrennung und neue deutsche rechtschreibung

\usepackage[dvips]{graphicx}
\usepackage{array}		% extrarowheight of tables
\usepackage{graphicx}   % eps import	
\usepackage[savemem,final]{listings}	% code listings
\usepackage{fancyhdr}	% for nicer headers
\usepackage{caption}
\usepackage{color}
\usepackage{url}
\usepackage{marvosym}	% for the stop symbol warning about important things
\usepackage{mparhack}				% notifies about changes in marginpar - layout
\usepackage{xspace}	% variable space after e.g. abbreviations
\usepackage{here}
\usepackage{lettrine}

% global settings
\pagestyle{headings}  
\graphicspath{{gfx/}}
%\DeclareGraphicsExtensions{.eps,.pdf,.png}
%\DeclareGraphicsRule{.eps}{eps}{.eps}{}
%\DeclareGraphicsRule{.pdf}{pdf}{.pdf}{}
%\DeclareGraphicsRule{.png}{png}{.png}{}


\usepackage{lmodern} %Type1-font for non-english texts and characters


%% Packages for Graphics & Figures %%%%%%%%%%%%%%%%%%%%%%%%%%
\usepackage{graphicx} %%For loading graphic files
%\usepackage{subfig} %%Subfigures inside a figure
%\usepackage{tikz} %%Generate vector graphics from within LaTeX

%% Please note:
%% Images can be included using \includegraphics{filename}
%% resp. using the dialog in the Insert menu.
%% 
%% The mode "LaTeX => PDF" allows the following formats:
%%   .jpg  .png  .pdf  .mps
%% 
%% The modes "LaTeX => DVI", "LaTeX => PS" und "LaTeX => PS => PDF"
%% allow the following formats:
%%   .eps  .ps  .bmp  .pict  .pntg


%% Math Packages %%%%%%%%%%%%%%%%%%%%%%%%%%%%%%%%%%%%%%%%%%%%
\usepackage{amsmath}
\usepackage{amsthm}
\usepackage{amsfonts}


%% Line Spacing %%%%%%%%%%%%%%%%%%%%%%%%%%%%%%%%%%%%%%%%%%%%%
%\usepackage{setspace}
%\singlespacing        %% 1-spacing (default)
%\onehalfspacing       %% 1,5-spacing
%\doublespacing        %% 2-spacing


%% Other Packages %%%%%%%%%%%%%%%%%%%%%%%%%%%%%%%%%%%%%%%%%%%
%\usepackage{a4wide} %%Smaller margins = more text per page.
%\usepackage{fancyhdr} %%Fancy headings
%\usepackage{longtable} %%For tables, that exceed one page


%%%%%%%%%%%%%%%%%%%%%%%%%%%%%%%%%%%%%%%%%%%%%%%%%%%%%%%%%%%%%
%% Remarks
%%%%%%%%%%%%%%%%%%%%%%%%%%%%%%%%%%%%%%%%%%%%%%%%%%%%%%%%%%%%%
%
% TODO:
% 1. Edit the used packages and their options (see above).
% 2. If you want, add a BibTeX-File to the project
%    (e.g., 'literature.bib').
% 3. Happy TeXing!
%
%%%%%%%%%%%%%%%%%%%%%%%%%%%%%%%%%%%%%%%%%%%%%%%%%%%%%%%%%%%%%

%%%%%%%%%%%%%%%%%%%%%%%%%%%%%%%%%%%%%%%%%%%%%%%%%%%%%%%%%%%%%
%% Options / Modifications
%%%%%%%%%%%%%%%%%%%%%%%%%%%%%%%%%%%%%%%%%%%%%%%%%%%%%%%%%%%%%

%\input{options} %You need a file 'options.tex' for this
%% ==> TeXnicCenter supplies some possible option files
%% ==> with its templates (File | New from Template...).



%%%%%%%%%%%%%%%%%%%%%%%%%%%%%%%%%%%%%%%%%%%%%%%%%%%%%%%%%%%%%
%% DOCUMENT
%%%%%%%%%%%%%%%%%%%%%%%%%%%%%%%%%%%%%%%%%%%%%%%%%%%%%%%%%%%%%
\begin{document}

\begin{titlepage}
\pagestyle{empty}

%\hfill\includegraphics[height=0.4cm]{}
%\vspace{-35mm}\includegraphics[height=0.4cm]{}
\begin{minipage}[h]{50mm}
\vspace{-10mm}
\hspace{-17mm}
%\includegraphics*[height=12mm]{Logo}	% todo better logo
\end{minipage}
%
\begin{minipage}[h]{80mm}
\vspace{-10mm}
\hspace{70mm}
%\includegraphics*[height=11mm]{2nd-Logo}
\end{minipage}

   \begin{center}
       \vspace*{2cm}
       \Huge
       \textsc{The TIGL Geometry Library}

       \vspace{0.5cm}
       \Large
       \textsc{German Aerospace Center\\\vspace{0.3cm}\small{Simulation and Software Technology (SC)\\Distributed Systems and Component Software}}

       \vspace{0.5cm}
       \large
       Version 2012\,/\,11

       \vspace{0.5cm}
       \vspace{1cm}
       \textsc{}
       \vspace{1cm}

                %\includegraphics[width=5cm, height=5cm]{Titelbild}
      

       \vspace{1cm}
       \Large
       \textsc{}
         \end{center}   
        
   \vfill
   \normalsize
   \begin{tabular}{ll}
       Authors: & Arne Bachmann, Markus Litz\\
                & Martin Siggel, Markus Kunde \\
       Date:   & \today\\
   \end{tabular}

\begin{minipage}[h]{80mm}
\vspace{-15mm}
\hspace{90mm}
\includegraphics*[height=1.1cm]{dlrlogo}
\end{minipage}

\end{titlepage}



%% Title Page %%%%%%%%%%%%%%%%%%%%%%%%%%%%%%%%%%%%%%%%%%%%%%%
%% ==> Write your text here or include other files.

%% The nice version:
%\input{titlepage} %%You need a file 'titlepage.tex' for this.
%% ==> TeXnicCenter supplies a possible titlepage file
%% ==> with its templates (File | New from Template...).


%% Inhaltsverzeichnis %%%%%%%%%%%%%%%%%%%%%%%%%%%%%%%%%%%%%%%
\tableofcontents %Table of contents
\cleardoublepage %The first chapter should start on an odd page.

\pagestyle{plain} %Now display headings: headings / fancy / ...



%% Chapters %%%%%%%%%%%%%%%%%%%%%%%%%%%%%%%%%%%%%%%%%%%%%%%%%
%% ==> Write your text here or include other files.


\chapter{The TIGL Geometry Library}\label{tiglIntro}

\section{What is TIGL}\label{tiglIntro}
In order to perform the modeling of wings and fuselages as well as the computation of surface points effectively, a
geometry library was developed in C++. The library provides external interfaces
for C and FORTRAN. Some of the requirements of the library were:

\begin{itemize}
 \item Ability to read and process the information stored in a CPACS file for
 wings and fuselages,
 \item Possibility to extend to engine pods, landing gear and other
 geometrical characteristics, e.g.,
 \item Ability to build up the three-dimensional airplane geometry for further
 processing,
 \item Ability to compute surface points in cartesian coordinates by using
 parameters such as $\xi$, $\zeta$, $\eta$, segment number,
 \item Possibility to be expanded by additional functions such as area or volume
 computations,
 \item Possibility to export the airplane geometry in the IGES format.
\end{itemize}


The developed library uses the Open Source software OpenCASCADE to represent the airplane geometry by B-spline surfaces
in order to compute surface points and also to export the geometry in the IGES format.
OpenCASCADE is a development platform written in C++ for CAD, CAM, and CAE
applications which has been continuously developed for more than ten years. 
The functionality covers geometrical primitives (for example points,
vectors, matrix operations), the computation of B-spline surfaces and boolean operations on volume models.

Apart from the already specified requirements above, the geometry library 
offers query functions for the geometry structure. These functions can be used
for example to detect how many segments are attached to a certain segment,
which indices these segments have, or how many wings and fuselages the current
airplane configuration contains. This functionality is necessary because not
only the modeling of simple wings or fuselages but also the description of quite complicated
structures with branches or flaps is targeted.


\section{TIGLCreator}
In order to review the geometry information of the central data set a visualization
tool, TIGLCreator, was developed. The TIGLCreator allows the visualization of the used airfoils and
fuselage profiles as well as of the surfaces and the entire airplane model.
Furthermore, the TIGLCreator can be used to validate and test the implemented
functions of the geometry library, for example the calculation of points on the
surface or other functions to check data that belong to the geometry structure.










 


\chapter{Installation}\label{hints}

First you need to install the most recent version of Open Cascade to your computer. You could find a direct download link in the CPACSIntegration-Teamsite: 

\url{http://sites.kp.dlr.de/sc/CPACSIntegration/default.aspx}

Please be sure to answer "`yes"' when asked if the open cascade setup should set the environment variables. 
The \textit{\$PATH} variable is extended and a new environment variable \textit{\$CASROOT} is created which points to the OpenCASCADE installation directory. 

Since OpenCASCADE 6.3.0 there is no longer a installer for Linux. For most distributions exist native installation packages (*.deb or *.rpm format).






% Erkl�ren das 


%The TIXI XML Interface is available for Linux and MS-Windows platforms. Supported compilers are gcc/g77 and on Win32 additionally Mircosoft Visual C++ 7.1 (optional with Intel Visual Fortran).
%
%
%\section{Getting TIXI}\label{gettingTIXI}
%The most recent version of TIXI could be downloaded from the CPACSIntegration Teamsite:
%
%\url{http://sites.kp.dlr.de/sc/CPACSIntegration/Freigegebene\%20Dokumente/Forms/AllItems.aspx}
%
%The downloaded TIXI zip file contains TIXI as shared and static library. Also the necessary include files and 3rd-party libraries are included. 
%
%
%
%\section{Installation}
%Extract the TIXI zip archive to your local hard drive and set the environment variable \emph{PATH} to the \emph{$TIXI$/lib} directory. 
%
%\emph{Hint:} On Microsoft Windows you will find the environment variables under: \textit{"`Systemsteuerung - System - Erweitert - Umgebungsvariablen"'}
%
%
%
%
%
 


\chapter{Using TIGL}\label{usingTIGL}

\section{Usage Example}\label{Usage Example}

\subsection{Open a CPACS configuration}
This is how to open a CPACS configuration. Please mention that first \emph{TIXI} has to be used to open a CPACS configuration.


\subsection{Querying the geometrical structure}
Below you find some example how to query the geometrical structure of a CPACS configuration.


\subsection{Example how to use TIGL out of python scripts}
This is a small sample script that does nothing more than opening an CPACS data set and exporting it as an iges file.

\begin{verbatim}
from ctypes import *
from os import *

# define handles
cpacsHandle = c_int(0)
tixiHandle = c_int(0)
filename   = "./cpacs_example.xml"
exportName = "./cpacs_example.iges"

# open TIXI and TIGL shared libraries
import sys
if sys.platform == 'win32':
    TIXI = cdll.TIXI
    TIGL = cdll.TIGL
else:
    TIXI = CDLL("libTIXI.so")
    TIGL = CDLL("libTIGL.so")

# Open a CPACS configuration file. First open the CPACS-XML file
# with TIXI to get a tixi handle and then use this handle to open
# and read the CPACS configuration.
tixiReturn = TIXI.tixiOpenDocument(filename, byref(tixiHandle))
if tixiReturn != 0:
    print 'Error: tixiOpenDocument failed for file: ' + filename
    exit(1)

tiglReturn = TIGL.tiglOpenCPACSConfiguration(tixiHandle, "VFW-614", byref(cpacsHandle))
if tiglReturn != 0:
    TIXI.tixiCloseDocument(tixiHandle)
    print "Error: tiglOpenCPACSConfiguration failed for file: " + filename
    exit(1)

#------------------------------------
# Export CPACS geometry as IGES file.
#------------------------------------
print "Exporting CPACS geometry as IGES file..."
TIGL.tiglExportIGES(cpacsHandle, exportName)

\end{verbatim}







%%%%%%%%%%%%%%%%%%%%%%%%%%%%%%%%%%%%%%%%%%%%%%%%%%%%%%%%%%%%%
%% BIBLIOGRAPHY AND OTHER LISTS
%%%%%%%%%%%%%%%%%%%%%%%%%%%%%%%%%%%%%%%%%%%%%%%%%%%%%%%%%%%%%
%% A small distance to the other stuff in the table of contents (toc)
\addtocontents{toc}{\protect\vspace*{\baselineskip}}

%% The Bibliography
%% ==> You need a file 'literature.bib' for this.
%% ==> You need to run BibTeX for this (Project | Properties... | Uses BibTeX)
%\addcontentsline{toc}{chapter}{Bibliography} %'Bibliography' into toc
%\nocite{*} %Even non-cited BibTeX-Entries will be shown.
%\bibliographystyle{alpha} %Style of Bibliography: plain / apalike / amsalpha / ...
%\bibliography{literature} %You need a file 'literature.bib' for this.

%% The List of Figures
%\clearpage
%\addcontentsline{toc}{chapter}{List of Figures}
%\listoffigures

%% The List of Tables
%\clearpage
%\addcontentsline{toc}{chapter}{List of Tables}
%\listoftables


%%%%%%%%%%%%%%%%%%%%%%%%%%%%%%%%%%%%%%%%%%%%%%%%%%%%%%%%%%%%%
%% APPENDICES
%%%%%%%%%%%%%%%%%%%%%%%%%%%%%%%%%%%%%%%%%%%%%%%%%%%%%%%%%%%%%
\appendix
%% ==> Write your text here or include other files.

%\input{FileName} %You need a file 'FileName.tex' for this.


\end{document}

