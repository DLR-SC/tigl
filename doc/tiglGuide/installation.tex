
\chapter{Installation}\label{hints}

First you need to install the most recent version of Open Cascade to your computer. You could find a direct download link in the CPACSIntegration-Teamsite: 

\url{http://sites.kp.dlr.de/sc/CPACSIntegration/default.aspx}

Please be sure to answer "`yes"' when asked if the open cascade setup should set the environment variables. 
The \textit{\$PATH} variable is extended and a new environment variable \textit{\$CASROOT} is created which points to the OpenCASCADE installation directory. 

Since OpenCASCADE 6.3.0 there is no longer a installer for Linux. For most distributions exist native installation packages (*.deb or *.rpm format).






% Erkl�ren das 


%The TIXI XML Interface is available for Linux and MS-Windows platforms. Supported compilers are gcc/g77 and on Win32 additionally Mircosoft Visual C++ 7.1 (optional with Intel Visual Fortran).
%
%
%\section{Getting TIXI}\label{gettingTIXI}
%The most recent version of TIXI could be downloaded from the CPACSIntegration Teamsite:
%
%\url{http://sites.kp.dlr.de/sc/CPACSIntegration/Freigegebene\%20Dokumente/Forms/AllItems.aspx}
%
%The downloaded TIXI zip file contains TIXI as shared and static library. Also the necessary include files and 3rd-party libraries are included. 
%
%
%
%\section{Installation}
%Extract the TIXI zip archive to your local hard drive and set the environment variable \emph{PATH} to the \emph{$TIXI$/lib} directory. 
%
%\emph{Hint:} On Microsoft Windows you will find the environment variables under: \textit{"`Systemsteuerung - System - Erweitert - Umgebungsvariablen"'}
%
%
%
%
%
