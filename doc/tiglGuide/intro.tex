
\chapter{The TIGL Geometry Library}\label{tiglIntro}

\section{What is TIGL}\label{tiglIntro}
In order to perform the modeling of wings and fuselages as well as the computation of surface points effectively, a
geometry library was developed in C++. The library provides external interfaces
for C and FORTRAN. Some of the requirements of the library were:

\begin{itemize}
 \item Ability to read and process the information stored in a CPACS file for
 wings and fuselages,
 \item Possibility to extend to engine pods, landing gear and other
 geometrical characteristics, e.g.,
 \item Ability to build up the three-dimensional airplane geometry for further
 processing,
 \item Ability to compute surface points in cartesian coordinates by using
 parameters such as $\xi$, $\zeta$, $\eta$, segment number,
 \item Possibility to be expanded by additional functions such as area or volume
 computations,
 \item Possibility to export the airplane geometry in the IGES format.
\end{itemize}


The developed library uses the Open Source software OpenCASCADE to represent the airplane geometry by B-spline surfaces
in order to compute surface points and also to export the geometry in the IGES format.
OpenCASCADE is a development platform written in C++ for CAD, CAM, and CAE
applications which has been continuously developed for more than ten years. 
The functionality covers geometrical primitives (for example points,
vectors, matrix operations), the computation of B-spline surfaces and boolean operations on volume models.

Apart from the already specified requirements above, the geometry library 
offers query functions for the geometry structure. These functions can be used
for example to detect how many segments are attached to a certain segment,
which indices these segments have, or how many wings and fuselages the current
airplane configuration contains. This functionality is necessary because not
only the modeling of simple wings or fuselages but also the description of quite complicated
structures with branches or flaps is targeted.


\section{TIGLViewer}
In order to review the geometry information of the central data set a visualization
tool, TIGLViewer, was developed. The TIGLViewer allows the visualization of the used airfoils and
fuselage profiles as well as of the surfaces and the entire airplane model.
Furthermore, the TIGLViewer can be used to validate and test the implemented
functions of the geometry library, for example the calculation of points on the
surface or other functions to check data that belong to the geometry structure.










