\section{Component segment coordinate transform}

\begin{figure}[htb]
  \centering
  % trim left, bottom, right , top
  \includegraphics[trim=1cm 1cm 1cm 0.5cm, clip=true,width = 8cm]{gfx/compSegTheo}
	\caption{Illustration for the calculation of the component segment coordinate transformation}
	\label{fig:cs_surf}
\end{figure}

Definitions:
\begin{itemize}
	\item Leading edge:  $ \vec s_v = \vec p_2 - \vec p_1 $
	\item Trailing edge: $ \vec s_h = \vec p_4 - \vec p_3 $
	\item Projected leading edge:   $\vec n = -\vec s_v$, with $n_x = 0$  
\end{itemize}


\subsection{Extending leading and trailing edges}
Plane through $\vec p_4$ with normal vector $\vec n$. Calculate inersection with leading edge. Plane equation is:
\begin{equation}
(\vec p - \vec p_4) \cdot \vec n = 0
\label{eq:plane}
\end{equation}
 Linear equation for the leading edge:
 
\begin{equation}
\vec p = \vec p_1 + \alpha (\vec p_2 - \vec p_1)
\label{eq:lin_eq1}
\end{equation}

Inserting (\ref{eq:lin_eq1}) into (\ref{eq:plane}) yields:
\begin{equation}
\alpha_v = \frac {(\vec p_4 - \vec p_1) \cdot \vec n }{(\vec p_2 - \vec p_1) \cdot \vec n}
\label{eq:nothing}
\end{equation}
%
If $\alpha_v > 1$, the leading edge has to be extended, else the trailing edge must be extended. In the first case, we calculate the extended leading edge point $\vec p_2^\prime$:
\begin{equation}
\vec p_2^\prime = \vec p_1 + \alpha_v (\vec p_2 - \vec p_1)
\label{eq:}
\end{equation}
%
In the other case ($\alpha_v < 1$), the intersection point with the trailing edge can be calculated in the same fashion.  \par
Now, we apply the same method also to the inner section, getting the extended points $\vec p_1^\prime$ and $\vec p_2^\prime$.

\subsection{Calculating $\eta$ values of the corners}
For the following calculations, we need to know the eta coordinates of the corner points $\vec p_1 \dots \vec p_4$. Lets image a plane with the previously defined normal vector $\vec n$ that goes through one of these points. Without loss of generality, let this point be $\vec p_3$. This plane is then defined by the equation
%
\begin{equation}
(\vec p - \vec p_3) \cdot \vec n = 0
\label{eq:plane_p3}
\end{equation}
%
Now we should find out, at which eta coordinate the plane intersects the leading edge, which is now parametrized as follows:
\begin{equation}
p = \vec p_1^ \prime + \eta ({\vec p_2}^\prime - {\vec p_1}^\prime).
\end{equation}
Combining both equations leads to $\eta_3 = \frac {(\vec p_3 - {\vec p_1}^\prime) \cdot \vec n }{({\vec p_2}^\prime - {\vec p_1}^\prime) \cdot \vec n}$, or in general:
\begin{equation}
\eta_i = \frac {(\vec p_i - {\vec p_1}^\prime) \cdot \vec n }{({\vec p_2}^\prime - {\vec p_1}^\prime)\cdot \vec n}.
\end{equation}

